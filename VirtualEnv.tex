\documentclass{article}
% https://github.com/Devinterview-io/docker-interview-questions
% Language setting
% Replace `english' with e.g. `spanish' to change the document language
\usepackage[english]{babel}


% Set page size and margins
% Replace `letterpaper' with `a4paper' for UK/EU standard size
\usepackage[letterpaper,top=2cm,bottom=2cm,left=3cm,right=3cm,marginparwidth=1.75cm]{geometry}
\usepackage[most]{tcolorbox}

\newtcolorbox{mybox}{
enhanced,
boxrule=0pt,frame hidden,
borderline west={4pt}{0pt}{green!75!black},
colback=green!10!white,
sharp corners
}
% Useful packages
\usepackage{amsmath}
\usepackage{graphicx, tcolorbox}
\usepackage[colorlinks=true, allcolors=blue]{hyperref}
\usepackage{xcolor}
\usepackage[dvipsnames]{xcolor}

\title{Virtual Environments}
\author{SS}

\begin{document}
\maketitle

\begin{abstract}
This article explains need of Virtual Environments and basic steps of implementation.
\end{abstract}

\section{Introduction}
When we run a Python file, and we have to import different libraries such as pandas, myplotlib etc.. The \hl{requests} library is a popular third party. The python file goes to a specified address/path in the OS to see the installation of those files.  \\
% 
%*********************************************************************
%*********************************************************************
\noindent
{\color{red} \rule{\linewidth}{0.5mm}}
\textcolor{red}{What is a Virtual Environment?} \\
\noindent
{\color{red} \rule{\linewidth}{0.5mm}}
This is simply a self-contained locations that enables you to maintain separate and isloated environments for your Python projects.  
\\
\textbf{Advantages} This isolation helps you manage
\begin{itemize}
\color{blue}
\item \textbf{Manage Dependencies}
\item \textbf{Versions}
\item \textbf{Packages}
\end{itemize}
without conflict across different projects.
\\


%*********************************************************************
%                      create and activate a Virtual Environment
%*********************************************************************
\noindent
{\color{red} \rule{\linewidth}{0.5mm}}
\textcolor{red}{How do you create and activate a Virtual Environment?} \\
\noindent
{\color{red} \rule{\linewidth}{0.5mm}}
\begin{itemize}
    \item Navigate to the directory where you want to create the virtual environment, the location of your python project.
    \item Here, create a new environment using the built in "venv" command
    \item There are other tools - conda, pipenv, virtualenv and poetry, can also do this.
    \item venv is built in python and is the most common one.
    \item The commands for creating and activating a virtual env are slightly different in mac and windows.
    \item On windows: python -m venv env
    \item \textlangle env \textrangle is the name of the environment. You can name this anything you like. It is convention to name it env.
    \item on Mac/Linus, the command is: python3 -m venev <env>
    \item At this point, it will create a new directory for us.
    \item Now, we need to activate the virtual environment.
    \item On window: \textlangle env\textrangle \textbackslash Scripts\textbackslash activate.bat
    \item if it does not work, then: \textlangle env \textrangle \textbackslash Scripts \textbackslash activate
    \item On Mac/Linux: source \textlangle env \textrangle \textbackslash bin \textbackslash activate
    \item Once you activate, you should see the prefix of the name of the environment at the begining of the command line.
\end{itemize}
%*********************************************************************
%                      deactivate a Virtual Environment
%*********************************************************************
\noindent
{\color{red} \rule{\linewidth}{0.5mm}}
\textcolor{red}{How do you deactivate a Virtual Environment?} \\
\noindent
{\color{red} \rule{\linewidth}{0.5mm}}
\begin{itemize}
    \item Type: deactivate
\end{itemize}
%*********************************************************************
%*********************************************************************
%*********************************************************************
% Reference
%*********************************************************************
\section{Reference}
\begin{itemize}
    \item \href{https://www.youtube.com/watch?v=Y21OR1OPC9As}{Tech With Tim}
\end{itemize}
\end{document}